% diagrams-Bd.tex
\begin{hcarentry}[updated]{diagrams}
\report{Brent Yorgey}%05/18
\status{active development}
\participants{many}
\makeheader

The diagrams framework provides an embedded domain-specific language for
declarative drawing. The overall vision is for diagrams to become a viable
alternative to DSLs like MetaPost or Asymptote, but with the advantages of
being \emph{declarative}---describing what to draw, not how to draw it---and
\emph{embedded}---putting the entire power of Haskell (and Hackage) at the
service of diagram creation. There is always more to be done, but diagrams is
already quite fully-featured, with a comprehensive user manual and a growing
set of tutorials, a large collection of primitive shapes and attributes, many
different modes of composition, paths, cubic splines, images, text, arbitrary
monoidal annotations, named subdiagrams, and more.

%**<img width=700 src="./3d-cubes.jpg">
%*ignore
\begin{center}
\includegraphics[width=0.7\columnwidth]{html/3d-cubes.jpg}
\end{center}
%*endignore

\subsubsection*{What's new}

Work on diagrams has slowed considerably since the release of diagrams
1.4 in October 2016, due to time constraints of the main developers.
However, work on diagrams 2.0 is slowly but steadily progressing,
targeting a release during the summer of 2018.  Updates will include:

\begin{compactitem}
\item Completely rewritten support for animations, with much better
  semantics and updated examples and tutorials.
\item Death to the type-level ``backend token'', which will allow much
  easier creation of diagrams that simultaneously work with multiple
  backends.
\item A complete rewrite of the library internals, resulting in better
  performance and enabling cool new features like diagram traversals.
\item Lots of small updates and improvements.
\end{compactitem}

%**<img width=700 src="./kaleidoscope.jpg">
%*ignore
\begin{center}
\includegraphics[width=0.7\columnwidth]{html/kaleidoscope.jpg}
\end{center}
%*endignore

\subsubsection*{Contributing}

There is plenty of exciting work to be done; new contributors are welcome!
Diagrams has developed an encouraging, responsive, and fun developer
community, and makes for a great opportunity to learn and hack on some
``real-world'' Haskell code. Because of its size, generality, and enthusiastic
embrace of advanced type system features, diagrams can be intimidating to
would-be users and contributors; however, we are actively working on new
documentation and resources to help combat this. For more information on ways
to contribute and how to get started, see the Contributing page on the
diagrams wiki: \url{http://haskell.org/haskellwiki/Diagrams/Contributing}, or
come hang out in the \texttt{\#diagrams} IRC channel on freenode.

%**<img width=700 src="./arrows.jpg">
%*ignore
\begin{center}
\includegraphics[width=0.7\columnwidth]{html/arrows.jpg}
\end{center}
%*endignore

\FurtherReading
\begin{compactitem}
\item \url{http://diagrams.github.io}
\item \url{http://diagrams.github.io/gallery.html}
\item \url{http://haskell.org/haskellwiki/Diagrams}
\item \url{http://github.com/diagrams}
\item \url{http://ozark.hendrix.edu/~yorgey/pub/monoid-pearl.pdf}
\item \url{http://www.youtube.com/watch?v=X-8NCkD2vOw}
\end{compactitem}
\end{hcarentry}
